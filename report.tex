\documentclass[a4paper,11pt]{bxjsarticle}
\usepackage{xltxtra}
\usepackage{zxjatype}
\usepackage{here}
\setjamainfont[BoldFont=ipaexm.ttf]{ipaexm.ttf}
\setjasansfont[BoldFont=ipaexg.ttf]{ipaexg.ttf}
\usepackage{enumitem}
\usepackage{amsmath}
\usepackage{amssymb}
\usepackage{booktabs}
\usepackage{listings}
\usepackage{bm}
\usepackage{cite} 
\newtheorem{mf}{公式}

\lstset{
basicstyle={\scriptsize}
}
\setlist[enumerate]{listparindent=2pt}


\begin{document}
\begin{titlepage}
  \begin{center}
    \vspace*{150truept}
    {\Huge 微分方程式}\\ % タイトル
    \vspace{120truept}
    {\huge 18T1694W}\\ % 学籍番号
    \vspace{50truept}
    {\huge 島袋 隆也}\\ % 著者
    \vspace{50truept}
    {\huge \today}\\ % 提出日
  \end{center}
\end{titlepage}


%=====================================================================
\section{目的}
 与えられた微分方程式の解を導出し,その式がどのような動作をするのかを考察することで
理解を深めることを目的とする

%=====================================================================
\section{公式}
計算に使う公式を以下に示す.
\begin{mf}
  $\frac{1}{D+\alpha}r(x) &= e{-\alpha x}\int r(x)e{\alpha x}$
  \label{mf:1}
\end{mf}
\begin{mf}
  $y_p=\frac{1}{P(D)}e^{\alpha x} &= \frac{1}{P(\alpha)}e^{\alpha x}, (P(\alpha)\neq 0)$
  \label{mf:2}
\end{mf}

          

%=====================================================================
\section{大問1}
問題文を以下に示す.
\begin{equation}
  y''-2y'-3y=e^x
  \label{equ:1}
\end{equation}
\begin{equation}
  y''-2y'-3y=e^{3x}
  \label{equ:2}
\end{equation}
\begin{equation}
  y''-2y'-3y=e^x+e^{3x}+\sin x
  \label{equ:3}
\end{equation}


\subsection{問1|$y''-2y'-3y=e^x$}
式(\ref{equ:1})は以下の特徴をもつ.

\begin{itemize}
  \item $y''$が最も高い階数→2階
  \item $y$の微分係数に$x$がかかっていない → 定数係数
  \item $y''$にかかる係数が1 → 正規系
  \item $y$と$y$の微分がすべて1乗 → 線形
  \item $e^x$が存在 → 非斉次
\end{itemize}

上記により(\ref{equ:1})式は線形非斉次微分方程式であることがわかる.\\

線形非斉次微分方程式において,演算子法を用いた特解の導出は以下の手順で行う.

\begin{enumerate}
  \item 演算子法を用いて表す.
  \item yについて解く.
  \item 変形した逆演算子を用いる.
\end{enumerate}

では,特解$y_p$を導出していく.\\
(\ref{equ:1})式を微分演算子を用い,$y_p$の式に直すと以下のようになる.

\begin{equation}
  y_p = \frac{1}{(D^2-2D-3)}e^x
\end{equation}

ここで,$(P(1)=-4\neq 0)$が成り立つため,公式(\ref{mf:2})を用いることができる.
よって特解$y_p$は

\begin{equation}
  y_p = -\frac{1}{4}e^{x}
\end{equation}

\subsection{問2|$y''-2y'-3y=e^{3x}$}
問2も問1と同様に線形非斉次微分方程式である.
よって,微分演算子を用い,因数分解し,$y_p$の式に直すと以下のようになる.

\begin{equation}
  y_p = \frac{1}{(D-1)(D-2)}e^{3x}
\end{equation}

ここで,$(P(3)= 0)$より公式(\ref{mf:2})が使えない.
そこで,基本公式である公式(\ref{mf:1})を用いる.
よって,特解$y_p$は

\begin{equation}
  \frac{1}{D-2}e^{3x}=e^{3x}
\end{equation}

となり,特解は
\begin{align}
  y_p &= \frac{1}{D-1}e^{3x}\\
  &= -\frac{1}{2}e^{3x}
\end{align}

\subsection{問3|$y''-2y'-3y=e^x+e^{3x}+\sin x$}

\section{大問2}
\subsection{}


%=====================================================================

\bibliography{webDB.bib}
\bibliographystyle{ieeetr} 



\end{document}
